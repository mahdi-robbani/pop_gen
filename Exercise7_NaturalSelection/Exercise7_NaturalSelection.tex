\PassOptionsToPackage{unicode=true}{hyperref} % options for packages loaded elsewhere
\PassOptionsToPackage{hyphens}{url}
%
\documentclass[]{article}
\usepackage{lmodern}
\usepackage{amssymb,amsmath}
\usepackage{ifxetex,ifluatex}
\usepackage{fixltx2e} % provides \textsubscript
\ifnum 0\ifxetex 1\fi\ifluatex 1\fi=0 % if pdftex
  \usepackage[T1]{fontenc}
  \usepackage[utf8]{inputenc}
  \usepackage{textcomp} % provides euro and other symbols
\else % if luatex or xelatex
  \usepackage{unicode-math}
  \defaultfontfeatures{Ligatures=TeX,Scale=MatchLowercase}
\fi
% use upquote if available, for straight quotes in verbatim environments
\IfFileExists{upquote.sty}{\usepackage{upquote}}{}
% use microtype if available
\IfFileExists{microtype.sty}{%
\usepackage[]{microtype}
\UseMicrotypeSet[protrusion]{basicmath} % disable protrusion for tt fonts
}{}
\IfFileExists{parskip.sty}{%
\usepackage{parskip}
}{% else
\setlength{\parindent}{0pt}
\setlength{\parskip}{6pt plus 2pt minus 1pt}
}
\usepackage{hyperref}
\hypersetup{
            pdftitle={Exercise 7 - Natural Selection},
            pdfauthor={Mahdi Robbani},
            pdfborder={0 0 0},
            breaklinks=true}
\urlstyle{same}  % don't use monospace font for urls
\usepackage[margin=1in]{geometry}
\usepackage{longtable,booktabs}
% Fix footnotes in tables (requires footnote package)
\IfFileExists{footnote.sty}{\usepackage{footnote}\makesavenoteenv{longtable}}{}
\usepackage{graphicx,grffile}
\makeatletter
\def\maxwidth{\ifdim\Gin@nat@width>\linewidth\linewidth\else\Gin@nat@width\fi}
\def\maxheight{\ifdim\Gin@nat@height>\textheight\textheight\else\Gin@nat@height\fi}
\makeatother
% Scale images if necessary, so that they will not overflow the page
% margins by default, and it is still possible to overwrite the defaults
% using explicit options in \includegraphics[width, height, ...]{}
\setkeys{Gin}{width=\maxwidth,height=\maxheight,keepaspectratio}
\setlength{\emergencystretch}{3em}  % prevent overfull lines
\providecommand{\tightlist}{%
  \setlength{\itemsep}{0pt}\setlength{\parskip}{0pt}}
\setcounter{secnumdepth}{0}
% Redefines (sub)paragraphs to behave more like sections
\ifx\paragraph\undefined\else
\let\oldparagraph\paragraph
\renewcommand{\paragraph}[1]{\oldparagraph{#1}\mbox{}}
\fi
\ifx\subparagraph\undefined\else
\let\oldsubparagraph\subparagraph
\renewcommand{\subparagraph}[1]{\oldsubparagraph{#1}\mbox{}}
\fi

% set default figure placement to htbp
\makeatletter
\def\fps@figure{htbp}
\makeatother


\title{Exercise 7 - Natural Selection}
\author{Mahdi Robbani}
\date{March 8, 2020}

\begin{document}
\maketitle

\hypertarget{natural-selection-and-other-complications}{%
\section{Natural selection (and other
complications)}\label{natural-selection-and-other-complications}}

\textbf{Hans R. Siegismund}

\hypertarget{exercise-1}{%
\subsubsection{Exercise 1}\label{exercise-1}}

In Africa and southern Europe, many human populations are polymorphic at
the locus coding for the beta-hemoglobin chain. Two alleles are found,
HbS, and HbA. HbS differs from HbA in that it at the position 6 the
amino acid glutamic acid has been replaced with valine. A study in
Tanzania found the following genotypic distribution:

\begin{longtable}[]{@{}lcccl@{}}
\toprule
& HbAHbA & HbAHbS & HbSHbS & Sum\tabularnewline
\midrule
\endhead
Adults & 400 & 249 & 5 & 654\tabularnewline
Children & 189 & 89 & 9 & 287\tabularnewline
\bottomrule
\end{longtable}

\begin{enumerate}
\def\labelenumi{\arabic{enumi})}
\item
  \textbf{Estimate the allele frequencies in both groups.}
\item
  \textbf{Do the observed genotype distributions differ from
  Hardy-Weinberg proportions?}
\end{enumerate}

\textbf{Answer:}

\begin{longtable}[]{@{}lllll@{}}
\toprule
Adults & HbAHbA & HbAHbS & HbSHbS & Sum\tabularnewline
\midrule
\endhead
Observed & 400 & 249 & 5 & 654\tabularnewline
Expected & 420.64 & 207.71 & 25.64 & 654\tabularnewline
\bottomrule
\end{longtable}

where

\emph{p}(A) = 0.802

\emph{p}(S) = 0.198

The test for Hardy-Weinberg proportions gives χ2 = 25.84, which is
highly significant. We see that this is caused by a large excess of
heterozygotes.

\begin{longtable}[]{@{}lllll@{}}
\toprule
Children & HbAHbA & HbAHbS & HbSHbS & Sum\tabularnewline
\midrule
\endhead
Observed & 189 & 89 & 9 & 287\tabularnewline
Expected & 189.97 & 87.05 & 9.97 & 287\tabularnewline
\bottomrule
\end{longtable}

where

\emph{p}(A) = 0.814

\emph{p}(S) = 0.186

In this case the test for Hardy-Weinberg proportions is χ2 = 0.14, which
is non-significant. We also see that the allele frequencies among
children and adults are similar. We do not bother to make a formal test
for it.

As might be known, the low survival of the HbSHbS genotype is due to
sickle cell anemia. They have a high probability of dying because of
this. The HbAHbS heterozygote is compared to the HbAHbA homozygote more
resistant to malaria. We therefore have a system of overdominance.

\begin{enumerate}
\def\labelenumi{\arabic{enumi})}
\setcounter{enumi}{2}
\tightlist
\item
  \textbf{Under the assumption that the polymorphism has reached a
  stable equilibrium, estimate the fitness of the three genotypes.
  Hints: Assume that the adults had a genotypic distribution equal to
  their expected Hardy-Weinberg distribution and estimate their relative
  fitness.}
\end{enumerate}

\textbf{Answer:}

\begin{longtable}[]{@{}lllll@{}}
\toprule
Adults & HbAHbA & HbAHbS & HbSHbS & Sum\tabularnewline
\midrule
\endhead
Observed & 400 & 249 & 5 & 654\tabularnewline
Expected & 420.64 & 207.71 & 25.64 & 654\tabularnewline
Fitness (O/E) & 0.951 & 1.199 & 0.195 &\tabularnewline
Relative fitness & 0.793 & 1.000 & 0.163 &\tabularnewline
Selection coefficient & 0.207 & 0 & 0.837 &\tabularnewline
\bottomrule
\end{longtable}

We can see that the selection against the HbSHbS genotype is very
severe. We can use the selection coefficient to estimate the equilibrium
allele frequency

\emph{p} = \emph{t}/(\emph{s} + \emph{t}) = 0.837/(0.207 + 0.837) =
0.802

This value is identical to the estimated allele frequency among the
adults. The reason is that we have estimated it under the assumption of
equilibrium.

In African Americans, one out of 400 suffers of sickle cell anemia.

\begin{enumerate}
\def\labelenumi{\arabic{enumi})}
\setcounter{enumi}{3}
\tightlist
\item
  \textbf{Estimate the allele frequencies among them}
\end{enumerate}

\textbf{Answer:}

\emph{q} = √(1/400) = 0.05

\begin{enumerate}
\def\labelenumi{\arabic{enumi})}
\setcounter{enumi}{4}
\tightlist
\item
  \textbf{Why is it lower than the frequency observed among Africans?}
\end{enumerate}

\textbf{Answer:}

There is no longer malaria in the USA. It has been eliminated.
Therefore, there is no longer overdominant selection that keeps a high
allele frequency of the deleterious allele. There has been directional
selection against the deleterious allele, which has reduced its
frequency.

\hypertarget{exercise-2}{%
\subsubsection{Exercise 2}\label{exercise-2}}

The figure to the right shows the result of 13 repeated experiments with
a chromosomal polymorphism in \emph{Drosophila meanogaster}. It shows
the frequency of one of the two chromosomal forms through four
generations the experiment lasted. In six of the experiments one type
had frequencies slightly higher than 0.9 and 7 of the experiments had
levels below 0.9. Population sizes in each experiment were 100.

\begin{enumerate}
\def\labelenumi{\arabic{enumi})}
\tightlist
\item
  \textbf{Can the evolution of this system be explained as a result of
  genetic drift?}
\end{enumerate}

\textbf{Answer:}

No.~It is highly unlikely that genetic drift in a population with a size
of 100 can result in fixation after 4 generations. In addition, all six
experiments starting with a frequency of about 0.9 end up being fixed
for one allele, while the seven experiments, which start with a rate is
below 0.9 all end up being fixed for the other allele. Genetic drift
would have a more ``random'' nature.

\begin{enumerate}
\def\labelenumi{\arabic{enumi})}
\setcounter{enumi}{1}
\tightlist
\item
  \textbf{Can the evolution be explained as a result of natural
  selection? How does it work? Which genotype has the lowest fitness?}
\end{enumerate}

\textbf{Answer:}

Yes; there must be underdominance where the heterozygote has a lower
fitness than both homozygotes.

\hypertarget{exercise-3}{%
\subsubsection{Exercise 3}\label{exercise-3}}

A geneticist starts an experiment with \emph{Drosophila melanogaster}.
He uses 10 populations, each kept at a constant size of 8 males and 8
females in each generation. In generation 0 all individuals are
heterozygous for the two alleles \emph{A}1 and \emph{A}2 at an autosomal
locus. After 19 generations, the following distribution of the allele
frequency of \emph{A}1 is observed in the 10 populations:

\begin{longtable}[]{@{}llllllllll@{}}
\toprule
0.18 & 0.00 & 0.18 & 0.25 & 0.30 & 0.19 & 0.16 & 0.00 & 0.15 &
0.00\tabularnewline
\midrule
\endhead
\bottomrule
\end{longtable}

\begin{enumerate}
\def\labelenumi{\arabic{enumi})}
\tightlist
\item
  \textbf{Can this distribution of the allele frequency be explained by
  genetic drift?}
\end{enumerate}

** Answer: **

No.~With genetic drift, the allele frequencies would be distributed
randomly over the entire range from 0 to 1. Here, all 10 populations
have an allele frequency of less than 0.5, which is very unlikely.
(0.510 = 0.000977)

\begin{enumerate}
\def\labelenumi{\arabic{enumi})}
\setcounter{enumi}{1}
\tightlist
\item
  \textbf{Which other evolutionary force has also worked during this
  experiment? }
\end{enumerate}

\textbf{Answer:}

Natural selection.

\emph{After 100 generations, all ten populations were fixed for allele
A}2\emph{.}

\begin{enumerate}
\def\labelenumi{\arabic{enumi})}
\setcounter{enumi}{2}
\tightlist
\item
  \textbf{Use this information to explain how the fitness of the three
  genotypes, \emph{w}11, \emph{w}12 and \emph{w}22, are related to each
  other.}
\end{enumerate}

\textbf{Answer:}

Natural selection has also been involved, in this case in the form of
directional selection, where

\emph{w}11 \textless{} \emph{w}12 \textless{} \emph{w}22,

\hypertarget{exercise-4}{%
\subsubsection{Exercise 4}\label{exercise-4}}

After the arrival of the Europeans in America, the California condor
(\emph{Gymnogyps californianus}) was severely hunted. This resulted in a
drastic decline in population size, which culminated in 1987 when the
last wild condors were placed in captivity (fourteen individuals).
{[}Later on, the condor was released again. In 2014, 425 were living in
the wild or in captivity.{]} Among the progeny of these fourteen
individuals the genetic disease chondrodystrophy (a form of dwarfism was
observed). In condor, this disease is inherited at an autosomal locus
where chondrodystrophy is due to a recessive lethal allele.

\begin{enumerate}
\def\labelenumi{\arabic{enumi})}
\tightlist
\item
  \textbf{What has the frequency of the allele for chondrodystrophy at
  least been among the fourteen individuals who were used to found the
  population in captivity?}
\end{enumerate}

\textbf{Answer:}

\emph{q} ≥ 2/(2 x 14) = 0.071

There must at least have been two heterozygotes among the fourteen
individuals.

\emph{The population has since grown in number and reached around a few
hundred. An estimation of the allele frequency for chondrodystrophy
showed a value of 0.09.}

\begin{enumerate}
\def\labelenumi{\arabic{enumi})}
\setcounter{enumi}{1}
\tightlist
\item
  \textbf{Can the frequency of this lethal allele be caused by one of
  the following three forces separately? (In question c you will be
  asked whether a combination of these forces is needed to explain the
  frequency.)}
\end{enumerate}

\begin{itemize}
\tightlist
\item
  mutation
\item
  genetic drift
\item
  natural selection
\end{itemize}

\textbf{Answer:}

\begin{itemize}
\tightlist
\item
  mutation: No
\item
  genetic drift: Yes
\item
  natural selection: No
\end{itemize}

\begin{enumerate}
\def\labelenumi{\arabic{enumi})}
\setcounter{enumi}{2}
\tightlist
\item
  \textbf{Is it necessary to consider that two or three of these forces
  act together to explain the frequency of this allele?}
\end{enumerate}

\textbf{Answer:}

The allele for chondrodystophy arose through mutation and genetic drift
has resulted in the high frequency. (Natural selection eliminates this
allele, and thus would not be able to explain the high frequency of
allele.)

\begin{enumerate}
\def\labelenumi{\arabic{enumi})}
\setcounter{enumi}{3}
\tightlist
\item
  \textbf{What is the expected frequency of the allele after a balance
  between mutation and selection? The mutation rate can be assumed to be
  μ = 10-6?}
\end{enumerate}

\textbf{Answer:}

The equilibrium between mutation and natural selection is given by

\emph{p} = √(μ/\emph{s}) = √(10-6 /1) = 0.001.

\hypertarget{exercise-5}{%
\subsubsection{Exercise 5}\label{exercise-5}}

Cystic fibrosis is caused by a recessive allele at a single autosomal
locus, CTFR (cystic fibrosis transmembrane conductance regulator). In
European populations 1 out of 2500 newborn children are homozygous for
the recessive allele.

\begin{enumerate}
\def\labelenumi{\arabic{enumi})}
\tightlist
\item
  \textbf{What is the frequency of the recessive allele in these
  populations?}
\end{enumerate}

\textbf{Answer:}

\emph{q} = √ (1/2500) = 1/50 = 0.02

\begin{enumerate}
\def\labelenumi{\arabic{enumi})}
\setcounter{enumi}{1}
\tightlist
\item
  \textbf{What fraction of all possible parental combinations has a
  probability of ¼ for having a child which is homozygous for the
  recessive allele?}
\end{enumerate}

\textbf{Answer:}

It must be the combination heterozygote × heterozygote:

2\emph{pq} × 2\emph{pq} (2 × 0.98 × 0.02)2 = 0.03922 = 0.0015

\emph{The disease used to be fatal during childhood if it is not
treated. Therefore, it must be assumed that the fitness of the recessive
homozygote must have been 0 during the main part of the human
evolutionary history.}

\begin{enumerate}
\def\labelenumi{\arabic{enumi})}
\setcounter{enumi}{2}
\tightlist
\item
  \textbf{Estimate the mutation rate, assuming equilibrium between the
  mutation and selection.}
\end{enumerate}

\textbf{Answer:}

\emph{q =} √\emph{(μ/s}),

where \emph{μ} in is the mutation rate and \emph{s} is the selection
coefficient, which must be 1 since the fitness is 0.

Therefore,

\emph{μ} = \emph{q}2\emph{s} = 0.022 × 1 = 0.0004

\emph{A direct estimate of the mutation rate was 6.7 × 10-7, which is
considerably lower than the estimate found in question c.}

\begin{enumerate}
\def\labelenumi{\arabic{enumi})}
\setcounter{enumi}{3}
\tightlist
\item
  \textbf{Which mechanism(s) may explain the high frequency of the
  recessive deleterious allele?}
\end{enumerate}

\textbf{Answer:}

It could be overdominant selection where the fitness of heterozygous
carriers is higher than in homozygous normal. There have been several
hypotheses for this: increased resistance against tuberculosis or
cholera has been suggested but there are no hard data to explain it.

\end{document}
